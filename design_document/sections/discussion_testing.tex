The overall test coverage generated from our IDE is 42\% of classes, with 25\% lines covered. In our data folder it has a test coverage of 100\% of classes, 25\% lines covered. Our manager classes have 100\% test coverage with 32\% lines covered. GUI controller classes does not have any unit tests written for them nor does any part of the GUI element of our app. This is due to the fact that it is very difficult to automate testing of these classes, as there is just no viable or efficient way to test them. GUI greatly impacts our test coverage as GUI accounts for a large part of the code base in the system, thus these classes being not tested, greatly affects our unit test coverage as a whole. Without the GUI, our test coverage would increase by a substantial amount. Some of the reference classes currently does not have any unit tests associated with them. This is solely because of the methods in these classes are mainly setters, and therefore does not provide a good reason to be tested. \\ \\All acceptance tests had a positive result, and produced a good outcome. All acceptance tests are fully working, however some of the acceptance test were not fully implemented, and this is because were more on the GUI stuff like importing and exporting XML. Another reason was because the feature for the given acceptance test were not implemented and thus could not be tested. Most of the manual testing passed, and the reason why not all of them passed was due to unimplemented features missing from the application, and therefore some tests were not carried out. Along with this, we also tested the jar executabile file on different platforms such as Linux, MacOS, and Windows operating systems. By doing manual testing on different platforms, we increase our certainty that the application will work.\\